\section{Metodologia}

A elaboração do método matemático que descreve corretamente o fenômeno físico é de extrema importância para o sucesso do projeto. 
As equações que governam a dinâmica do escoamento sanguíneo numa artéria coronária serão desenvolvidas segundo a hipótese do meio contínuo. 
Dessa forma, os príncipios de conservação de massa, de quantidade de movimento linear e de espécie química poderão ser utilizados. 
O sangue será considerado como um fluido incompressível, newtoniano e multifásico, 
como também o coeficiente difusivo será aproximado como constante. 
A equação de Navier-Stokes será apresentada segundo as variáveis primitivas (velocidade-pressão) com 
a equação de transporte de espécie química sem geração interna
usando uma abordagem Lagrangeana-Euleriana Arbitrária (ALE) \cite{donea2004}.

\medskip
O desenvolvimento computacional será feito em linguagem C++/Python \cite{c++} \cite{python} 
utilizando o paradigma de orientação a objetos com objetivo de reaproveitar o código em outras pesquisas e 
as equações de governo serão discretizadas em cima de uma malha tetraédrica não estruturada gerada através do software livre GMSH \cite{gmsh}. 
Devido a restrição Babuska-Brezzi \cite{babuska1971} \cite{brezzi1974}, serão utiliados elementos distintos para os campos de velocidade e de pressão. 
As equações serão discretizadas no tempo utilizando a expansão da série de Taylor e o Método semi-Lagrangeano \cite{pironneau1982} será usado com o intuito de reduzir as oscilações espúrias que são características das equações do tipo convecção-difusão.
A formulação de Galerkin \cite{zienkiewicz1965} será utilizada para discretizarmos as equações no espaço. 

\medskip
Para solução do sistema de equações lineares oriundo da utilização do método de elementos finitos, 
deseja-se utilizar as técnicas mais modernas disponiveis em biblioteca pública de cálculo numérico. 
O emprego destas técnicas permitirá a utilização de diversos tipos de precondicionadores e 
métodos iterativos para uma eficiente solução do problema linear. 
A visualização da solução numérica encontrada através da solução do sistema linear será realizada através do software livre PARAVIEW \cite{paraview}, 
possibilitando alto nível de detalhamento da soluções numérica, através de cortes geométricos, interpolações e vetorização de campos.
