\section{Resumo}

A simulação numérica é uma ferramenta importante para solucionar problemas encontrados em diversos
processos físicos, como na extração de petróleo, enchimento de reservatórios, arrefecimento de
componentes eletrônicos e caracterização da hidrodinâmica em sistemas biológicos, relacionados a
problemas decorrentes de doença arterial coronariana (DAC). Este último é de grande importância devido
o enorme fardo econômico na sociedade. Este plano de trabalho tem como objetivo desenvolver uma
estrutura computacional para simular o escoamento em uma artéria coronária em coordenadas cartesianas.
O Método dos Elementos Finitos (MEF) será aplicado para resolver as equações de governo do
escoamento sanguíneo na artéria coronária com aterosclerose e stent farmacológico implantado. O sangue
será modelado como um fluido multifásico, incompressível e não-newtoniano. A equação de consevação de movimento linear e a equação de transporte de espécie química serão apresentadas de acordo com a descrição Lagrangeana-Euleriana Arbitrária (ALE). O método semi-Lagrangeano será usado a fim de reduzir as oscilações espúrias que podem ser observadas
quando o termo convectivo é predominante.
