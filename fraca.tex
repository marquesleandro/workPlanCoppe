\subsection{Formulação Fraca}

O restultado da ponderação da equação de
governo sobre o domínio é conhecida como
\textbf{formulação fraca} \cite{anjos2007}.
A seguir será explanada a formulação fraca.
Para mais detalhes, consultar \cite{brenner1994}.
Como o objetivo é encontrar uma solução
aproximada, é aceitável supor que seja produzido
um \textbf{Resíduo R} na equação de governo,
isto é:

\begin{equation}
 \frac{d^2 u}{dx^2} - \textit{Pe} \frac{du}{dx} = R
\end{equation}

Buscaremos forçar o resíduo ser equivalente
a zero num sentido médio \cite{finlayson1972}, logo:

\begin{equation}
 \int_{\Omega} Rwd\Omega = 0
\end{equation}


onde w é função peso. A função peso
é um conjunto de funções arbitrárias
dentro de um espaço de funções que será
discutido à frente. Possuímos, então,
as seguintes integrais:

\begin{equation}
 \int_{\Omega} \Bigg[\frac{d^2 u}{dx^2} - \textit{Pe} \frac{du}{dx} \Bigg] wd\Omega = 0
\end{equation}

Desenvolvendo a integral, temos:

\begin{equation}
 \int_{\Omega} \frac{d^2 u}{dx^2}wd\Omega - \textit{Pe}\int_{\Omega} \frac{du}{dx}wd\Omega = 0
\end{equation}

No termo difusivo, aplicaremos a integração
por partes com o intuito de diminuir a ordem
da derivada. Assim temos:

\begin{equation} 
  w\frac{du}{dx} - \int_{\Omega} \frac{du}{dx} \frac{dw}{dx} d\Omega 
  - \textit{Pe}\int_{\Omega} \frac{du}{dx}wd\Omega = 0
\end{equation}

O primeiro termo da equação acima é conhecido
como condição natural. Quando \textit{x = 1},
\textit{w = 0} por hipótese. Assim, a equação será:

\begin{equation}
  - w(0)\frac{du}{dx}(0) - \int_{\Omega} \frac{du}{dx}  \frac{dw}{dx} d\Omega 
  - \textit{Pe}\int_{\Omega} \frac{du}{dx}wd\Omega = 0
\end{equation}

isto é:

\begin{equation}
 - w(0)[u(0) - 1] - \int_{\Omega} \frac{du}{dx} \frac{dw}{dx} d\Omega 
 - \textit{Pe}\int_{\Omega} \frac{du}{dx}wd\Omega = 0
\end{equation}

ou seja:

\begin{equation}
 - w(0)u(0) + w(0) - \int_{\Omega} \frac{du}{dx} \frac{dw}{dx} d\Omega 
 - \textit{Pe}\int_{\Omega} \frac{du}{dx}wd\Omega = 0
\end{equation}



Simplificando a equação temos:

\begin{equation}
 \begin{align}
  \textbf{k}(u,w) & = \int_\Omega \frac{du}{dx} \frac{dw}{dx} d\Omega + w(0)u(0) \\
  \textbf{c}(u,w) & = \int_\Omega w \frac{du}{dx} d\Omega \\
  \textbf{f}(u,w) & = - w(0) 
 \end{align}
\end{equation}

Desta forma, a equação pode ser apresentada como:

\begin{equation}
 \textbf{k}(u,w) + \textit{Pe}  \textbf{c}(u,w) + \textbf{f}(u,w) = 0
\end{equation}

Dado o espaço de funções bases:

\begin{equation}
 \mathbb{U} = \{u \in \Omega \rightarrow \mathbb{R}
 : \int_\Omega \frac {du}{dx}^2 d\Omega < \infty 
 ; u(1) = 0\}
\end{equation}

Dado o espaço de funções pesos:

\begin{equation}
 \mathbb{W} = \{w \in \Omega \rightarrow \mathbb{R}
 : \int_\Omega \frac {dw}{dx}^2 d\Omega < \infty 
 ; w(1) = 0\}
\end{equation}

Desta forma, a formulação fraca consiste em
encontrarmos a solução de $u \in \mathbb{U}$
tal que:

\begin{equation}
 \textbf{k}(u,w) + \textit{Pe}  \textbf{c}(u,w) + \textbf{f}(u,w) = 0
\end{equation}

para todo w \in \mathbb{W}.
