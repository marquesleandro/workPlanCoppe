\section{Objetivos}

Para a realização do projeto proposto, planeja-se desenvolvolver simulações através da construção de um algoritmo de Elementos Finitos para modelos com geometrias complexas da artéria coronária com aterosclerose e com stent farmacológico implantado. 
Com a implementação desse código, é necessária sua validação através de comparações com possíveis resultados analíticos para se garantir que o modelo é uma boa representação do fenômeno estudado. 
O trabalho realizado e seus resultados também serão apresentados com publicação em canais de comunicação de grande visibilidade e alto fator de impacto. 
Com isso, objetiva-se a descrição detalhada de escoamentos em uma artéria coronária.

\bigskip
Os indicadores de desempenho que serão utilizados no projeto estão baseados em publicações produzidas na COPPE-UFRJ e 
submetidos a avaliações da comunidade científica. 
Como objetivos específicos, espera-se realizar:

\begin{enumerate}
\item Desenvolvimento de um código em Elementos Finitos utilizando a descrição Lagrangeana-Euleriana Arbitrária (ALE) para a conservação de movimento linear e o transporte de espécie química para um fluido incompressível, multifásico e não-newtoniano;

\item Conhecer a dinâmica do escoamento sanguíneo numa artéria coronária com aterosclerose
e com stents farmacológico implantado;

\item Publicação dos resultados em canais de comunicação nacionais e internacionais de excelência e alto impacto;

\item Defesa de tese de Doutorado.

\end{enumerate}
