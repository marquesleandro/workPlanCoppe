\section{Relevância}

De acordo com a Organização Mundial da Saúde (OMS) \cite{oms}, mais pessoas morrem anualmente devido às doenças cardiovasculares (DCV) do que qualquer outra causa no mundo a cada ano. 
Estima-se que 17,7 milhões de pessoas morreram por DCV em 2015, representando 31\% de todas as mortes no mundo. 
Aproximadamente 40\% das mortes por DCV ocorreram devido às doenças na artéria coronária (DAC). 
A principal causa da DAC é a aterosclerose que consiste no acúmulo de placas de gordura no interior da parede da artéria ocasionando uma diminuição do diâmetro do lúmen. 
A aterosclerose pode ser prevenida com uma mudança de hábitos nocivos tais como: o uso de tabaco, o uso de álcool, falta de atividade física e dietas não saudáveis \cite{spring2013}. 
Para uma abordagem corretiva, porém, dois tratamentos podem ser realizados: o bypass coronário (também conhecido como ponte de safena) e a angioplastia coronária transluminal percutânea (PTCA). 
O PTCA é um procedimento minimamente invasivo onde um tubo aramado, chamado stents, é colocado \cite{sigwart1987}.

\medskip
Este trabalho, portanto, tem como principais objetivos o desenvolvimento de um código em Elementos Finitos utilizando a descrição Lagrangeana-Euleriana Arbitrária (ALE) \cite{donea2004} para a conservação de movimento linear e o transporte de espécie química para um fluido incompressível, multifásico e não-newtoniano, além de 
conhecer a dinâmica do escoamento sanguíneo numa artéria coronária com aterosclerose e com stents farmacológico implantado.

