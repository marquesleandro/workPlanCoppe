\section{Introdução}

De acordo com a Organização Mundial da Saúde (OMS) \cite{oms}, mais pessoas morrem anualmente devido às doenças cardiovasculares (DCV) do que qualquer outra causa no mundo a cada ano. 
Estima-se que 17,7 milhões de pessoas morreram por DCV em 2015, representando 31\% de todas as mortes no mundo. 
Aproximadamente 40\% das mortes por DCV ocorreram devido às doenças na artéria coronária (DAC). 
A principal causa da DAC é a aterosclerose que consiste no acúmulo de placas de gordura no interior da parede da artéria ocasionando uma diminuição do diâmetro do lúmen. 
A aterosclerose pode ser prevenida com uma mudança de hábitos nocivos tais como: o uso de tabaco, o uso de álcool, falta de atividade física e dietas não saudáveis \cite{spring2013}. 
Para uma abordagem corretiva, porém, dois tratamentos podem ser realizados: o bypass coronário (também conhecido como ponte de safena) e a angioplastia coronária transluminal percutânea (PTCA). 
O PTCA é um procedimento minimamente invasivo onde um tubo aramado, chamado stents, é colocado \cite{sigwart1987}.
Os principais objetivos deste plano de trabalho são o desenvolvimento de um código em Elementos Finitos utilizando a descrição Lagrangeana-Euleriana Arbitrária (ALE) \cite{donea2004} para a conservação de movimento linear e o transporte de espécie química para um fluido incompressível, multifásico e não-newtoniano, além de 
conhecer a dinâmica do escoamento sanguíneo numa artéria coronária com aterosclerose e com stents farmacológico implantado.

\bigskip
As equações que governam a dinâmica do escoamento sanguíneo numa artéria coronária serão desenvolvidas segundo a hipótese do meio contínuo. 
Dessa forma, os príncipios de conservação de massa, de quantidade de movimento linear e de espécie química poderão ser utilizados. 
O sangue será considerado como um fluido incompressível, não-newtoniano e multifásico, como também o coeficiente difusivo será aproximado como constante. 
 
\bigskip
O desenvolvimento computacional será feito em linguagem C++/Python \cite{c++} \cite{python} utilizando o paradigma de orientação a objetos com objetivo de reaproveitar o código em outras pesquisas e 
as equações de governo serão discretizadas em cima de uma malha tetraédrica não estruturada através do Método dos Elementos Finitos. 
As equações serão discretizadas no tempo utilizando a expansão da série de Taylor e 
o Método semi-Lagrangeano \cite{pironneau1982} será utilizado com o intuito de reduzir as oscilações espúrias que são características das equações do tipo convecção-difusão e 
a formulação de Galerkin \cite{zienkiewicz1965} será utilizada para discretizarmos as equações no espaço. 

\bigskip 
Este projeto visa ser desenvolvido no Laboratório de Transmissão e Tecnologia do Calor (LTTC) do Programa de Engenharia Mecânica (PEM/COPPE). 
Tal laboratório dispõe de uma ampla infraestrutura, 
além de recursos computacionais de última geração, 
voltada a projetos de pesquisa, desenvolvimento e inovação
em termociências, em diversas linhas, 
inclusive em simulações com o uso de métodos numéricos \cite{lttccoppe}. 


